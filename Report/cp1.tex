\section{Proposal}

\subsection{Survey}

This section is supposed to enumerate (at least) \emph{four} references and
briefly summarize the main insight they provided (1 paragraph each). All
references should also be included in the References section that concludes the
report.

\begin{packed_enum}
   \item \ldots
   \item \ldots
   \item \ldots
   \item \ldots
   \item Additional references \ldots
\end{packed_enum}

\subsection{System}

We chose \ldots as the basis for our project.

\paragraph{Rationale} Why is this type of system interesting to you? (1-2
paragraphs)

In the following, discuss \emph{four} interesting
features/capabilities/properties of the chosen system (1 paragraph each).
\begin{packed_enum}
    \item \ldots
    \item \ldots
    \item \ldots
    \item \ldots
    \item Additional features/capabilities/properties \ldots
\end{packed_enum}

\subsection{Application Description}

Describe your application scenario (2-3 paragraphs) and justify why this
fits your system of choice (1-2 paragraphs).

\paragraph{Architectural Overview}

Provide an overview on the planned architecture/pipeline of your application.
For example: Programming language(s), 3rd-party libraries/frameworks and their
role in your application pipeline (1 paragraph each), planned/desired
deployment, \ldots

\paragraph{Experimental Data}

For each dataset, describe the origin as well as the properties/contents.
Moreover, justify why the respective datasets suit your application (2-3
paragraphs each).

\subsection{Roadmap}

\emph{Optional}. Split your project into individual steps and provide a first
roadmap for the project/semester. You may use the \verb|vroadmap| environment
(see \verb|report.tex| for definition):

\begin{vroadmap}
  YYYY-mm-dd & Step 1 \tabularnewline
  YYYY-mm-dd & Step 2 \tabularnewline
  \ldots & \ldots \tabularnewline
\end{vroadmap}
